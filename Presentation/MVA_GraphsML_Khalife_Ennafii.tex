% This text is proprietary.
% It's a part of presentation made by myself.
% It may not used commercial.
% The noncommercial use such as private and study is free
% Nov. 2006
% Author: Sascha Frank 
% University Freiburg 
% www.informatik.uni-freiburg.de/~frank/
%
% additional usepackage{beamerthemeshadow} is used
%  
%  \beamersetuncovermixins{\opaqueness<1>{25}}{\opaqueness<2->{15}}
%  with this the elements which were coming soon were only hinted
\documentclass{beamer}
\usepackage{beamerthemeshadow}
\usepackage{graphicx}
\usepackage[utf8]{inputenc}  
\usepackage[T1]{fontenc} 
%\usepackage[top=2cm,bottom=2cm,left=2cm,right=2cm,asymmetric]{geometry}

\usepackage{tkz-graph}
\usepackage{amsmath}
\usepackage{array,multirow,makecell}
\usepackage{float}
\usepackage{cancel}
\usepackage{algorithm}
\usepackage{algpseudocode}
\usepackage{subfig}
\usepackage{wrapfig}
\usepackage{ stmaryrd }
\usepackage{placeins}
\usepackage{ amssymb }
\usepackage{mathtools}
\begin{document}
\title{Significant perturbations in observability graph\\}  
\author{Oussama Ennafii \& Sammy Khalife}
\institute{ENS Cachan - Master MVA}
\date{\today} 

\frame{\titlepage} 

%\frame{\frametitle{Table of contents}\tableofcontents} 


\section{Introduction} 

\subsection{Introduction}
\frame{ \frametitle{Introduction}
\begin{itemize}
	\item Presentation feedback graph
	\item Sequence of losses $(l_t)_{1 \leq t \leq T}$ and actions $(I_t)_{1 \leq t \leq T} $
	~\\
	~\\
	\item Minimax regret $ R(G,T)=\min_{S} \max_{l_1,...,l_T} \mathbb{E}  \sum_{t=1}^{T} l_t(I_t) -\min_{i} \sum_{t=1}^{T}l_t(i)  $
	~\\
\end{itemize}
~\\
		%\centering \includegraphics[width=5cm]{Afib_ecg.jpg}


}


\frame{\frametitle{Well-known configuration}
		\begin{itemize}
			\item EWF \qquad \qquad \qquad \qquad \qquad \qquad \qquad EXP3 configuration
			\end{itemize}
			\begin{tikzpicture}[->,>=stealth',shorten >=1pt,auto,node distance=3cm,
			thick,main node/.style={circle,draw,font=\Large\bfseries}]
			\node[main node] (1) at (0,0) {1};
			\node[main node] (2) at (-1.5,-0.7) {2};
			\node[main node] (3) at (-0.7, -2) {3};
			\node[main node] (4) at (0.7,-2) {4};
			\node[main node] (5) at (1.5,-0.7) {5};
			\path
			(1) edge [loop above] node {} (1);
			\path
			(2) edge [loop below] node {} (2);
			\path
			(3) edge [loop below] node {} (3);
			\path
			(4) edge [loop below] node {} (4);
			\path
			(5) edge [loop below] node {} (5);
			
			\tikzstyle{LabelStyle}=[fill=white,sloped]
			%\tikzstyle{EdgeStyle}=[bend left]
			\Edge[](1)(2)
			\Edge[](1)(3)
			\Edge[](1)(4)
			\Edge[](1)(5)
			\Edge[](2)(4)
			\Edge[](4)(3)
			\Edge[](5)(4)
			\Edge[](5)(3)
			%\tikzstyle{EdgeStyle}=[bend right]
			\Edge[](5)(1)
			\Edge[](4)(1)
			\Edge[](3)(2)
			\Edge[](4)(2)
			\Edge[](3)(4)
			\Edge[](4)(5)
			\Edge[](3)(1)
			\Edge[](3)(5)
			\Edge[](2)(1)
			\end{tikzpicture}
			\hfill
			\begin{tikzpicture}[->,>=stealth',shorten >=1pt,auto,node distance=3cm,
			thick,main node/.style={circle,draw,font=\Large\bfseries}]
			\node[main node] (1) at (0,0) {1};
			\node[main node] (2) at (-1.5,-0.7) {2};
			\node[main node] (3) at (-0.7, -2) {3};
			\node[main node] (4) at (0.7,-2) {4};
			\node[main node] (5) at (1.5,-0.7) {5};
			\path
			(1) edge [loop above] node {} (1);
			\path
			(2) edge [loop above] node {} (2);
			\path
			(3) edge [loop above] node {} (3);
			\path
			(4) edge [loop above] node {} (4);
			\path
			(5) edge [loop above] node {} (5);
			\end{tikzpicture}
}

\subsection{Previous results}
\frame{\frametitle{Previous results}
\begin{itemize}
 
 \item  Bounds on the minimax regret with respect of the geometry of the graph :
 ~\\
 ~\\
\item Strongly observable with independence number $\alpha$ : $ R(G,T)= \tilde{O}(\alpha^{1/2}T^{1/2})$.
\item Weakly observable with domination number $\delta$ : $ R(G,T) = \tilde{O}(\delta^{1/2}T^{1/2})$
 \item Unobservable : $ R(G,T) = \tilde{O}(T) $.
\end{itemize}
}




\end{document}